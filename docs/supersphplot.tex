\documentclass[a4paper,12pt]{article}
\usepackage{natbib,epsfig,rotating}

\title{Visualisation using SUPERSPHPLOT - v1.0}
\author{Daniel Price}

\begin{document}
\maketitle
\tableofcontents

\section{Introduction}
 Here is my attempt at a user guide.

\subsection{What is it?}
SUPERSPHPLOT is a utility for visualisation of output from simulations using the
Smoothed Particle Hydrodynamics (SPH) method. It is written entirely in FORTRAN
90 and utilises PGPLOT subroutines to do the actual plotting.

\subsection{Why would I want it?}
 It makes life a whole lot easier.

\section{Getting started}
 First of all, make sure that PGPLOT is installed on your system. The PGPLOT
 graphics subroutine library is
freely downloadable from \emph{http://www.astro.caltech.edu/~tjp/pgplot/} or by
ftp from \emph{ftp://ftp.astro.caltech.edu/pub/pgplot/pgplot5.2.tar.gz}. Consult
the PGPLOT userguide
\emph{http://www.astro.caltech.edu/~tjp/pgplot/contents.html} 

\subsection{Compiling the code: makefile options} 
 In the Makefile, you will need to set the FORTRAN compiler and flags to your local version, e.g..
\begin{verbatim}
F90C = f95
F90FLAGS = -O
\end{verbatim}
 Secondly the compiler must be able to link to the PGPLOT and X11 libraries on
your system. As a first attempt try using:
\begin{verbatim}
LDFLAGS = -lpgplot -lX11
\end{verbatim}
If that works at a first attempt, take a moment to think several happy thoughts about your system
administrator. If these libraries are not found, you will need to enter the
library paths by hand. On most systems this is something like:
\begin{verbatim}
LDFLAGS = -L/usr/local/pgplot -lpgplot -L/usr/X11R6/lib -lX11
\end{verbatim}
(assuming the PGPLOT libraries are in the /usr/local/pgplot directory and the
X11 libraries are in /usr/X11R6/lib). If, having found the PGPLOT and X11
libraries, the program still won't compile, it is usually
because the PGPLOT on your system has been compiled with a different compiler to
the one you are using. A first attempt is to try using the g2c libraries
\begin{verbatim}
LDFLAGS = -L/usr/local/pgplot -lpgplot -L/usr/X11R6/lib -lX11 -lg2c
\end{verbatim}
On some systems I have also had to use
\begin{verbatim}
LDFLAGS = -L/usr/local/pgplot -lpgplot -L/usr/X11R6/lib -lX11 -lg2c -lpng
\end{verbatim}
Failing that, ask your system administrator!!

\subsection{Reading data}
 The data format is specified in the subroutine read\_data.  
The filename is input on the command line, ie.
\begin{verbatim}
supersphplot myrun
\end{verbatim}
With multiple filenames on the command line, ie.
\begin{verbatim}
supersphplot myrun1 myrun2 myrun3
\end{verbatim}
or simply
\begin{verbatim}
supersphplot myrun*.dat
\end{verbatim}
files will be read consecutively in the order that they are given.

\section{Program structure}

\section{Calculating additional quantities}

\section{Options}

\subsection{Plot limits}
 The options for plot limits are as follows:
\begin{enumerate}
\item Adaptive limits : limits are minimum and maximum of quantities at current
timestep. However, the co-ordinate limits are not adapted in the case of
rendered plots.
\item Fixed limits: limits by default are
\item Particle tracking: limits are centred on a particular particle, with
offsets as input by the user.
\end{enumerate}

\subsection{Co-ordinate transformations}

\subsection{Rotation}

\section{Rendering/contour plots}
 For a contour or rendered plot of a scalar quantity $\phi$ we
interpolate from the particles to an array of pixels using the SPH summation
interpolant.

In two dimensions the interpolant is simply
\begin{equation}
\phi(x,y) = \sum_b m_b \frac{\phi_b}{\rho_b} W(x - x_b, y-y_b, h_b)
\end{equation}
where $W$ is the usual cubic spline kernel and the summation is over
contributing particles.

In three dimensions, we must either take a cross section or a projection
through the data.

\subsection{Cross sections}
 A cross section can be taken of SPH data by summing the
contributions to each pixel in the cross section plane from all particles within
$2h$ of the plane. This is done in the subroutine interpolate\_3D\_fastxsec. In
this case the cross section is always at a fixed value of the third co-ordinate
(ie. for xy plots the cross section is in the z direction). Oblique cross
sections can be taken by rotating the particles first.

 A routine is also provided to do cross sections of two dimensional data (ie. take one dimensional
cross sections) (interpolate2D\_xsec). In this case the cross-section line can be arbitrarily
specified. I will hopefully write an interactive version of this one day.


\subsection{Projections}


\section{Vector plots}

\section{Isosurface plots}

\section{Exact solutions}

\subsection{Shock tubes (Riemann problem)}
 This subroutine plots the exact solution for a one-dimensional shock tube
(Riemann problem). The difficult bit of the problem is to determine the jump in
pressure and velocity across the shock front given the initial left and right
states. This is performed in a separate subroutine (riemannsolver) as there are 
many different methods by which this can be done (see e.g. \citealt{toro92}). 
The actual subroutine exact\_shock reconstructs the shock profile (consisting of
a rarefaction fan, contact discontinuity and shock, summarised in Figure
\ref{fig:shocktube}), given the post-shock values of pressure and
velocity. 

 The speed at which the shock travels into the `right' fluid can be computed from the post shock
velocity using the relation
\begin{equation}
v_{shock} = v_{post}\frac{(\rho_{post}/\rho_R)}{(\rho_{post}/\rho_R)- 1},
\end{equation}
where the jump conditions imply
\begin{equation}
\frac{\rho_{post}}{\rho_R} = \frac{(P_{post}/P_R) + \beta}{1 + \beta (P_{post}/P_R)}
\end{equation}
with
\begin{equation}
\beta = \frac{\gamma - 1}{\gamma + 1}.
\end{equation}

\subsubsection{Riemann solver}
 The algorithm for determining the post-shock velocity and pressure is taken
from \citet{vanleer79} (reprinted as \citealt{vanleer99}).


\subsection{Polytrope}
 This subroutines computes the exact solution for a static polytrope with
arbitrary $\gamma$.

\subsection{Linear wave}
 This subroutine simply plots a sine function on a given graph.

\subsection{Sedov blast wave}
 This subroutine computes the self-similar Sedov solution for a blast wave.

\subsection{Toy stars}
 See \citet{mp04}. The system is one dimensional with velocity $v$, density $\rho$, and pressure
$P$. The acceleration equation is 
\begin{equation}
\frac{dv}{dt} = - \frac{1}{\rho} \frac{\partial P}{\partial x}  - \Omega^2 x,
\end{equation}
 We assume the equation of state is 
\begin{equation}
P = K \rho^\gamma,
\end{equation} 

 The exact solutions provided assume the equations are scaled such that
$\Omega^2 = 1$.
 
\subsubsection{Static structure}
The static structure is given by
\begin{equation}
\bar \rho = 1- x^2,
\end{equation}

\subsubsection{Linear solutions}
The linear solution for the velocity is given by
\begin{equation}
v = 0.05 C_s G_n(x) \cos{\omega t} )
\end{equation}
density is
\begin{equation}
\rho = \bar{\rho} + \eta
\end{equation}
where 
\begin{equation}
\eta = 0.1 C_s \omega P_{n+1}(x) \sin{(\omega t)})
\end{equation}

\subsubsection{Non-linear solution}
In this case the velocity is given by
\begin{equation}
v = A(t) x,
\end{equation}
whilst the density solution is
\begin{equation}
\rho^{\gamma -1} = H(t) - C(t) x^2.
\end{equation}
where the parameters A, H and C are determined by solving the ordinary
differential equations
\begin{eqnarray}
\dot{H} & = & -AH(\gamma -1), \\
\dot{A} & = & \frac{2K \gamma}{\gamma -1} C - 1 - A^2 \\
\dot{C} & = & -AC(1+ \gamma),
\end{eqnarray}
The relation
\begin{equation}
A^2 = -1 - \frac{2 \sigma C}{\gamma -1} + kC^{\frac{2}{\gamma +1}},
\label{eq:kconst}
\end{equation}
is used to check the quality of the solution of the differential equations by
evaluating the constant $k$ (which should remain close to its initial value).

\subsection{MHD shock tubes}
 These are some tabulated solutions for specific MHD shock tube problems at a
given time taken from the tables given in \citet{dw94} and \citet{rj95}.

\subsection{h vs $\rho$}
 The subroutine exact\_hrho simply plots the relation between smoothing length
and density, ie.
\begin{equation}
h = h_{fact} \left(\frac{m}{\rho}\right)^{1/\nu}
\end{equation}
where $\nu$ is the number of spatial dimensions. The parameter $h_{fact}$ is
output by the code into the header of each timestep. For particles of different
masses, a different curve is plotted for each different mass value.

\section{Wishlist for future improvements}

\section*{Acknowledgements}
 My knowledge of SPH is derived almost entirely from Joe Monaghan.

\bibliographystyle{klunamed}
\bibliography{/home/dprice/bibtex/sph}

\end{document}
