\documentclass[a4paper,12pt]{article}

\title{NDSPMHD user guide - v1.0}
\author{Daniel Price}

\begin{document}
\maketitle
\tableofcontents

\section{Introduction}
 Hello and welcome to NDSPMHD! (I'll think of a better name soon). I hope you
enjoy using this code and that it may bring you many happy hours of research.
This is a fairly crap user guide so far but maybe I'll get around to writing a
proper one some day. 

\subsection{What is it?}
NDSPMHD is a 1, 2 and 3 dimensional code which solves the equations of
compressible gas
dynamics numerically using the Smoothed Particle Hydrodynamics (SPH) method. It
is written entirely in FORTRAN 90.

\subsection{Code features}
 So what is so unique about this code? A number of things:
\begin{itemize}
\item The same code can be compiled to run in 1, 2 or 3 dimensions. Big whoops,
you say, but this means that (for example) all the SPH bits of the 3D code can be thoroughly tested
on simple one dimensional problems. Whenever new physics is implemented, this
means that it can be thoroughly tested in 1D and then generalised in the blink
of a compiler to become a full performance, 3D code.
\item Lots of physics. The main thing you get in this code is an implementation
of Magnetohydrodynamics and hopefully, eventually, general relativity.
\item High accuracy. Many SPH codes are, well, kind of crap. The viscosity can
be way too high which means that the results are always really smoothed out.
Inconsistency between equation sets can also lead to subtle errors in SPH
approximations. All of the equation sets in this code have been derived
self-consistently from variational principles, leading to good conservation
properties and good accuracy.
\item Lots of options, but sensible defaults. By default the code will just run,
and run well. However, there are lots of times in SPH when something might not
work as well as it should for the specific problem you are trying to run, so
there are lots of other ways of doing things. Or you might want to compare
results with someone else's code and so you want the settings to be the same to
see where your simulation differs. For example this code can be run either
evolving the thermal energy, the total energy (default) or the entropy.
Similarly the magnetic field variable can be either $\mathbf{B}$ or
$\mathbf{B}/\rho$ (default). By default correction terms for a variable smoothing length
are calculated self-consistently, but it is also possible to use the old method
of using an average smoothing length between particle pairs, or the average of
the kernels. The equations implement the most general form of the momentum
equation possible, so that many different forms of the momentum equation can be
used (however, this code ensures that the continuity and energy equations are
consistent with any alternative formulation of the momentum equation). 
\item Setup tools. Setting up complicated particle arrangements is made a lot
easier with a whole lot of setup tools which can be called by the user, to
arrange particles in cartesian, cylindrical and spherical geometries,
particularly where there are strange density setups.
\item Plotting tools. I supply a number of plotting tools specifically tailored
for the analysis of SPH data. These include rendering utilities which
interpolate from the particles to an array of pixels to give beautifully
rendered movies.
\end{itemize}

\subsubsection{What you don't get.}
\begin{itemize}
\item A tree code. At least not yet anyway. Which means no self-gravity.
\end{itemize}

\section{Getting started}

\subsection{Obtaining and installing the code}
The code can be obtained by email request to: dprice@ast.cam.ac.uk. You
should obtain a single, gzipped tar archive. To install the code, type
\begin{verbatim}
gunzip ndspmhd.tar.gz
tar xvf ndspmhd.tar
\end{verbatim}
Now move into the code directory:
\begin{verbatim}
cd ndspmhd
\end{verbatim}
and have a look. Typing
\begin{verbatim}
ls
\end{verbatim}
 you should see something like the following:
\begin{verbatim}
src/ crap/ crap/
\end{verbatim}
The actual code is located in the src/ directory. The other
directories are:
\begin{table}[!h]
\begin{tabular}{ll}
\hline
docs/ & contains the code documentation (such as this document) \\
src/ & source directory (actual code)\\
multi/ & tools for running multiple jobs\\
scripts/ & shell scripts to perform various tasks\\
plot/ & visualisation tool for particle plots using PGPLOT \\
evplot/ & visualisation tool for evolution plots (energy vs time etc) using
PGPLOT \\
\hline
\end{tabular}
\end{table}

\noindent Type
\begin{verbatim}
make install
\end{verbatim}
or simply `make' to compile the whole
code. 

\subsection{Compiling the code: makefile options}
For an introduction to Makefiles in general, see for example the GNU make manual
at: \textbf{http://www.gnu.org/software/make/manual/make.html}. The Makefile in
the root directory has
the following options:
\begin{table}[!h]
\begin{tabular}{ll}
\hline
make install & compiles 1,2 and 3D code and utilities \\
make 1D & compiles the code in 1D \\
make 2D & compiles the code in 2D \\
make 25D & compiles the code in 2.5D \\
make 3D & compiles the code in 3D \\
make plotsph & compiles the particle plotting utility \\
make plotev & compiles the evolution file plotting utility \\
make multirun & compiles the utility for doing multiple runs \\
make clean & removes all .o and .mod files from the source directory \\
make edit1D & edits the current 1D setup file \\
make edit2D & edits the current 2D setup file \\
make edit25D & edits the current 2.5D setup file \\
make edit3D & edits the current 3D setup file \\
make scripts & makes symbolic links to useful scripts in root directory \\
make make &  edits the Makefile in the source directory \\
make save & saves a version of the code \\
\hline
\end{tabular}
\end{table}

Files are edited using the application specified by the current value of the
EDITOR variable in the Makefile. For example, to use xemacs instead of emacs, change the line in
the Makefile to read
\begin{verbatim}
EDITOR = xemacs
\end{verbatim}

 The compilation of the code uses the Makefile in the source directory
(ndspmhd/src/Makefile). This is where the compiler options and setup files can
be changed.

\subsection{Running a job}
\subsubsection{Creating an input file}

\subsubsection{Rerunning a job from the last position}

\subsection{Code options}

\subsection{Test problems}
 A suite of test problems

\section{Setting up a simulation}

\subsection{Using the interactive setup}

\subsection{Writing your own particle setup}

\section{The Gory details}

\subsection{Neighbour finding}

\subsection{Time stepping}

\section{Common errors}

\section{Visualisation using supersphplot}

\section{Wishlist for future improvements}

\section*{Acknowledgements}

\end{document}
