\documentclass[a4paper,12pt]{article}
\usepackage{natbib,epsfig,rotating}

\title{NDSPMHD user guide - v1.0}
\author{Daniel Price}

\begin{document}
\maketitle
\tableofcontents

\section{Introduction}
 Hello and welcome to NDSPMHD! (I'll think of a better name soon). I hope you
enjoy using this code and that it may bring you many happy hours of research.
This is a fairly crap user guide so far but maybe I'll get around to writing a
proper one some day. 

\subsection{What is it?}
NDSPMHD is a 1, 2 and 3 dimensional code which solves the equations of
compressible gas
dynamics numerically using the Smoothed Particle Hydrodynamics (SPH) method. It
is written entirely in FORTRAN 90.

\subsection{Code features}
 So what is so unique about this code? A number of things:
\begin{itemize}
\item The same code can be compiled to run in 1, 2 or 3 dimensions. Big whoops,
you say, but this means that (for example) all the SPH bits of the 3D code can be thoroughly tested
on simple one dimensional problems. Whenever new physics is implemented, this
means that it can be thoroughly tested in 1D and then generalised in the blink
of a compiler to become a full performance, 3D code.
\item Lots of physics. The main thing you get in this code is an implementation
of Magnetohydrodynamics and hopefully, eventually, general relativity.
\item High accuracy. Many SPH codes are, well, kind of crap. The viscosity can
be way too high which means that the results are always really smoothed out.
Inconsistency between equation sets can also lead to subtle errors in SPH
approximations. All of the equation sets in this code have been derived
self-consistently from variational principles, leading to good conservation
properties and good accuracy.
\item Lots of options, but sensible defaults. By default the code will just run,
and run well. However, there are lots of times in SPH when something might not
work as well as it should for the specific problem you are trying to run, so
there are lots of other ways of doing things. Or you might want to compare
results with someone else's code and so you want the settings to be the same to
see where your simulation differs. For example this code can be run either
evolving the thermal energy, the total energy (default) or the entropy.
Similarly the magnetic field variable can be either $\mathbf{B}$ or
$\mathbf{B}/\rho$ (default). By default correction terms for a variable smoothing length
are calculated self-consistently, but it is also possible to use the old method
of using an average smoothing length between particle pairs, or the average of
the kernels. The equations implement the most general form of the momentum
equation possible, so that many different forms of the momentum equation can be
used (however, this code ensures that the continuity and energy equations are
consistent with any alternative formulation of the momentum equation). 
\item Setup tools. Setting up complicated particle arrangements is made a lot
easier with a whole lot of setup tools which can be called by the user, to
arrange particles in cartesian, cylindrical and spherical geometries,
particularly where there are strange density setups.
\item Plotting tools. I supply a number of plotting tools specifically tailored
for the analysis of SPH data. These include rendering utilities which
interpolate from the particles to an array of pixels to give beautifully
rendered movies.
\end{itemize}

\subsubsection{What you don't get.}
\begin{itemize}
\item A tree code. At least not yet anyway. Which means no self-gravity.
\end{itemize}

\section{Getting started}

\subsection{Obtaining and installing the code}
The code can be obtained by email request to: dprice@ast.cam.ac.uk. You
should obtain a single, gzipped tar archive. To install the code, type
\begin{verbatim}
gunzip ndspmhd.tar.gz
tar xvf ndspmhd.tar
\end{verbatim}
Now move into the code directory:
\begin{verbatim}
cd ndspmhd
\end{verbatim}
and have a look. Typing
\begin{verbatim}
ls
\end{verbatim}
 you should see something like the following:
\begin{verbatim}
src/ crap/ crap/
\end{verbatim}
The actual code is located in the src/ directory. The other
directories are:
\begin{table}[!h]
\begin{tabular}{ll}
\hline
docs/ & contains the code documentation (such as this document) \\
src/ & source directory (actual code)\\
multi/ & tools for running multiple jobs\\
scripts/ & shell scripts to perform various tasks\\
plot/ & visualisation tool for particle plots using PGPLOT \\
evplot/ & visualisation tool for evolution plots (energy vs time etc) using
PGPLOT \\
\hline
\end{tabular}
\end{table}

\noindent Type
\begin{verbatim}
make install
\end{verbatim}
or simply `make' to compile the whole
code. 

\subsection{Compiling the code: makefile options}
For an introduction to Makefiles in general, see for example the GNU make manual
at: \textbf{http://www.gnu.org/software/make/manual/make.html}. The Makefile in
the root directory has
the following options:
\begin{table}[!h]
\begin{tabular}{ll}
\hline
make install & compiles 1,2 and 3D code and utilities \\
make 1D & compiles the code in 1D \\
make 2D & compiles the code in 2D \\
make 25D & compiles the code in 2.5D \\
make 3D & compiles the code in 3D \\
make plotsph & compiles the utility for particle plots \\
make plotev & compiles the utility for evolution file plotting \\
make multirun & compiles the utility for doing multiple runs \\
make clean & removes all .o and .mod files from the source directory \\
make edit1D & edits the current 1D setup file \\
make edit2D & edits the current 2D setup file \\
make edit25D & edits the current 2.5D setup file \\
make edit3D & edits the current 3D setup file \\
make scripts & makes symbolic links to useful scripts in root directory \\
make make &  edits the Makefile in the source directory \\
make save & saves a version of the code \\
\hline
\end{tabular}
\end{table}

Files are edited using the application specified by the current value of the
EDITOR variable in the Makefile. For example, to use xemacs instead of emacs, change the line in
the Makefile to read
\begin{verbatim}
EDITOR = xemacs
\end{verbatim}

 The compilation of the code uses the Makefile in the source directory
(ndspmhd/src/Makefile). This is where the compiler options and setup files can
be changed.

\subsection{Running a job}
\subsubsection{Creating an input file}

\subsubsection{A quick guide to the input file options}
Or... what does this button do?

In this section we briefly describe the options available for each 
\begin{table}[!h]
\begin{tabular}{rrp{0.9\textwidth}}
\hline
tmax & & maximum run time for the code \\
tout & & time interval for dumps \\
nout & & number of timesteps interval for dumps (ignored if negative) \\
nmax & &maximum number of timesteps \\
gamma & - & ratio of specific heats $\gamma$ ($\gamma$=1 gives isothermal) \\
icty & - & choice of density evaluation \\
    & 0 & density by direct summation \\
    & $>$0 & evolve continuity equation (using the consistent form if
    iprterm is set) \\
iprterm & - & choice of pressure term \\
    & $<$ 0 & gives no pressure contribution to the force \\
    & 0 & (default) gives usual SPH pressure evaluation \\
    & 1 & gives alternative form with $\phi = \rho$ \\
    & 2 & gives \citet{hk89} alternative form, ie. $\phi = \sqrt{P}/\rho$ \\ 
iav & - & artificial viscosity type \\
    & 1 & gives \citet{monaghan97}-type viscosity \\
iavlim & - & \citet{mm97} artificial viscosity switch \\
       & 1 & (default) uses the source term max($-\nabla\cdot{\bf v}, 0.0$) \\
       & 2 & uses the source term max($-\nabla\cdot{\bf v}(2.0-\alpha_{min}), 0.0)$) \\
ikernav & - & specifies how to average the kernel \\
        & 1 & uses average smoothing length, $\bar{h} = \frac12(h_a + h_b)$\\
	& 2 & uses average kernel $\bar{\nabla_a W_{ab}} = \frac12[\nabla_a
	W_{ab}(h_a) + \nabla_a W_{ab}(h_b)]$ \\
	& 3 & (default) uses variable smoothing length symmetrised form \\
ihvar   & - & method of smoothing length update \\
        & 1 & obsolete \\
	& 2 & evolves h alongside the other equations \\
	& 3 & evolves h, but iterates with density if necessary (only if
	ikernav = 3) \\
hfact   & - & specifies initial smoothing length in units of the average particle spacing \\
        & & recommended is 1.1-1.5 - default hfact = 1.2  \\
	& & this option determines the number of neighbours, so $h>1.2$ is very expensive in 3D \\
idumpghost & - & include ghost particles in the output dump? \\
         & 0 & no \\
	 & 1 & yes \\
imhd     & - & turns magnetic field on/off and sets induction equation type \\
         & 0 & no magnetic field \\
	 & 1-10 & magnetic field evolution using ${\bf B}/\rho$ \\
	 & $>=$ 11 & magnetic field evolution using ${\bf B}$ \\
imagforce & - & magnetic force algorithm \\
          & 0 & no magnetic force \\
	  & 2 & (default) tensor magnetic force (momentum-conserving) \\
	  & 5 & approximate, but stable tensor force from \citet{morrisphd} \\
	  & 1 & vector (Jx B) force (obsolete) \\
idivbzero & - & divergence correction method \\
          & 0 & no divergence correction \\
	  & 1 & divergence correction by projection method \\
	  & $>$ 1 & hyperbolic/parabolic divergence cleaning \\
psidecayfact & - & controls strength of the parabolic divergence cleaning term \\
          & & similar to avdecayconst \\
ianticlump & - & anti-clumping artificial stress term (only relevant when imagforce = 2) \\
          & 0 & no artificial stress \\
	  & 1 & use artificial stress \\
eps       & 0.0-2.0 & value of $\epsilon$ in artificial stress term \\
neps      & 1-5 & power in anticlumping term \\
ixsph, xsphfac & & use XSPH, with parameter xsphfac \\
igravity  & & calculate self-gravity by direct summation \\
damp      & & damp the velocity by this fraction each timestep \\
          & 0.0 & no artificial damping  \\
	  & 0.01 & damp the velocity by $1\%$ per timestep \\
ikernel   & & choice of smoothing kernel \\
          & 0,1 & (default) cubic spline \\
	  & 2 & quartic spline \\
	  & 3 & quintic spline \\
	  & 4 & rescaled quintic spline \\
	  & $>$5 & some other kernels \\
	  & 10 & Gaussian truncated at $r=5h$ \\	  
\hline
\end{tabular}
\end{table}

\subsubsection{Rerunning a job from the last position}

\subsection{Using multirun to create multiple input files}
 The multirun program is given as a way of rapidly creating lots of different
input files, useful for example where a large number of runs are required with a simple
variation of a particular parameter.

Firstly, create an input file called 'multirun.in'. Do this by running ndspmhd
in any number of dimensions, e.g.
\begin{verbatim}
1DSPMHD multirun
\end{verbatim}
Then run the multirun program
\begin{verbatim}
./multi/multirun myrun 6
\end{verbatim}
where 6 is the number of runs. The input files are named sequentially as
myrun1.in, myrun2.in, myrun3.in etc., with options identical to those given in
multirun.in, except where modified in multirun.f90 (use the templates provided
to create your own version of this program to adjust the input parameters
between runs as required). To run the code on all jobs sequentially
type
\begin{verbatim}
1DSPMHD myrun*.in
\end{verbatim}
Note that this will only work if the code exits normally from all previous runs.
For a failsafe way of running multiple jobs, regardless of how they exit,  a
simple shell script can be used such as the domulti.pl script in the scripts
directory (domulti makes a new directory, invokes multirun to create the input
files and then runs the jobs in the new directory). Several example multirun.f90
programs are given for running the test suites (such as the MHD shocks).

\subsection{Code options}

\subsection{Test problems}
 A suite of test problems

\section{Setting up a simulation}

\subsection{Using the interactive setup}

\subsection{Writing your own particle setup}

\section{The Gory details}

\subsection{Kernel calculation}
 Several kernels are given as part of the code. The standard kernel is the cubic
 spline kernel \citep{ml85}, given by
\begin{equation}
f(q) = \sigma \left\{ \begin{array}{ll}
1 - \frac{3}{2}q^2 + \frac{3}{4}q^3, & 0 \le q < 1; \\
\frac{1}{4}(2-q)^3, & 1 \le q < 2; \\
0. & q \ge 2. \end{array} \right. \label{eq:cubicspline}
\end{equation}
with normalisation $\sigma = [2/3,10/(7\pi),1/\pi]$.
This kernel satisfies the basic requirements
(\ref{eq:kernelcondfirst})-(\ref{eq:kernelcondlast}), has continuous first
derivatives and compact support of size $2h$. Smoother kernels can be introduced
by increasing the size of the compact support region (which correspondingly
increases the cost of evaluation by increasing the number of contributing
neighbours) and by using higher order interpolating spline functions. To this end the
quartic spline kernel
\begin{equation}
f(q) = \sigma \left\{ \begin{array}{ll}
(2.5-q)^4 - 5(1.5-q)^4 + 10(0.5-q)^4, & 0 \le q < 0.5; \\
(2.5-q)^4 - 5(1.5-q)^4, & 0.5 \le q < 1.5; \\
(2.5-q)^4, & 1.5 \le q < 2.5; \\
0. & q \ge 2.5. \end{array} \right. \label{eq:quarticspline} 
\end{equation}
 with normalisation $\sigma = [1/24,96/1199\pi,1/20\pi]$ and quintic spline kernel
\begin{equation}
f(q) = \sigma \left\{ \begin{array}{ll}
(3-q)^5 - 6(2-q)^5 + 15(1-q)^5, & 0 \le q < 1; \\
(3-q)^5 - 6(2-q)^5, & 1 \le q < 2; \\
(3-q)^5, & 2 \le q < 3; \\
0. & q \ge 3. \end{array} \right. \label{eq:quinticspline}
\end{equation}
 with normalisation $\sigma = [1/120,7/478\pi,1/120\pi]$ can be used \citep[e.g.][]{morrisphd}. The higher order polynomials have the advantage of
smoother derivatives which, in combination with the increased size of compact
support, decreases the sensitivity of the kernel to disorder in the particle
distribution (\S\ref{sec:kernelstability}).

 The kernel is tabulated for a faster evaluation. 

\begin{figure}[h]
\begin{center}
\begin{turn}{270}\epsfig{file=kernels.ps,height=1.3\textwidth}\end{turn}
\caption{SPH smoothing kernels given in kernelND.f90 (solid
line) together with their first (dashed) and second (dot-dashed) derivatives. Kernels
correspond to those given in the text. The cubic spline (top left) is the usual
choice, whilst the quintic (top, middle) represents a closer approximation to the Gaussian
kernel (top right), at the cost of increased compact support.}
\label{fig:kernels}
\end{center}
\end{figure}

\subsection{Density evaluation}

\subsection{Physics}

\subsection{Neighbour finding}

\subsection{Time stepping}

\section{Common errors}

\section{Visualisation}

\subsection{Supersphplot}
 Supersphplot (located in the subdirectory plot) is a utility for visualising
the output contained in the .dat file created by ndspmhd which contains the main
dumps of particle data from the code. A separate user guide is provided.

\subsection{Evsupersph}
 Evsupersph (located in the subdirectory evplot) is a simple visualisation tool for the data contained in the
.ev file created by ndspmhd (this monitors the conservation and time evolution
of various quantities). For pure hydrodynamics, the .ev file monitors the
conservation of energy and momentum, whilst for MHD many additional quantities
are output, mostly error parameters related to any non-zero divergence of the
magnetic field.

To view a .ev file using evsupersph, use the following:
\begin{verbatim}
evsupersph myrun
\end{verbatim}
for comparison between various different runs, simply add the filenames to the
command line, ie:
\begin{verbatim}
evsupersph myrun1 myrun2 myrun3
\end{verbatim}
or simply
\begin{verbatim}
evsupersph myrun*.ev
\end{verbatim}
With multiple files, evsupersph will plot the data from each file using a
different line style and include a legend on the plot. Text for the legend is by
default the filename, but can be changed by creating a file called `legend', in
which each line contains the text for each legend line, for example:
\begin{verbatim}
echo first run > legend
echo second run >> legend
echo third run >> legend
\end{verbatim}
*** give an example of use of evsupersph on a run

\subsection{Kernel plotting}
 A utility for plotting the kernel used by the main code is provided in the
utils directory (kernelplot.f90). This can be useful to investigate alternative
kernels. This also calls a routine to plot the stability properties of a given
kernel for one-dimensional, isothermal SPH. 

\section{Wishlist for future improvements}

\section*{Acknowledgements}
 My knowledge of SPH is derived almost entirely from Joe Monaghan.

\bibliographystyle{klunamed}
\bibliography{/home/dprice/bibtex/sph}

\end{document}
