\documentclass[a4paper,12pt]{article}

\title{NDSPMHD user guide - v1.0}
\author{Daniel Price}

\begin{document}
\maketitle
\tableofcontents

\section{Introduction}
 Hello and welcome to NDSPMHD! (I'll think of a better name soon). I hope you
enjoy using this code and that it may bring you many happy hours of research.
This is a fairly crap user guide so far but maybe I'll get around to writing a
proper one some day. 

\subsection{What is it?}
NDSPMHD is a 1, 2 and 3 dimensional code which solves the equations of fluid
dynamics numerically using the Smoothed Particle Hydrodynamics (SPH) method. It
is written entirely in FORTRAN 90.

\subsection{Code features}
 So what is so unique about this code? A number of things:
\begin{itemize}
\item The same code can be compiled to run in 1, 2 or 3 dimensions. Big whoops,
you say, but this means that (for example) all the SPH bits of the 3D code can be thoroughly tested
on simple one dimensional problems. Whenever new physics is implemented, this
means that it can be thoroughly tested in 1D and then generalised in the blink
of a compiler to become a full performance, 3D code.
\item Lots of physics. The main thing you get in this code is an implementation
of Magnetohydrodynamics and hopefully, eventually, general relativity.
\item High accuracy. Many SPH codes are, well, kind of crap. The viscosity can
be way too high which means that the results are always really smoothed out.
Inconsistency between equation sets can also lead to subtle errors in SPH
approximations. All of the equation sets in this code have been derived
self-consistently from variational principles, leading to good conservation
properties and good accuracy.
\item Setup tools. Setting up complicated particle arrangements is made a lot
easier with a whole lot of setup tools which can be called by the user, to
arrange particles in cartesian, cylindrical and spherical geometries,
particularly where there are strange density setups.
\end{itemize}

\subsubsection{What you don't get.}
\begin{itemize}
\item A tree code. At least not yet anyway. Which means no gravity.
\end{itemize}

\section{Getting started}

\subsection{Obtaining and compiling the code}
The code can be obtained by email request to: dprice@ast.cam.ac.uk. You
should obtain a single, gzipped tar archive. To install the code, type
\begin{verbatim}
gunzip ndspmhd.tar.gz
tar xvf ndspmhd.tar
\end{verbatim}
Now move into the code directory:
\begin{verbatim}
cd ndspmhd
\end{verbatim}
and have a look. Typing
\begin{verbatim}
ls
\end{verbatim}
 you should see something like the following:
\begin{verbatim}
src/ crap/ crap/
\end{verbatim}
The actual code is located in the src/ directory. The other
directories are:
\begin{tabular}{ll}
docs & contains the code documentation (such as this document) \\
src & source directory (actual code)\\
multi & tools for running multiple jobs\\
scripts & shell scripts to perform various tasks\\
plot & visualisation tool for particle plots using PGPLOT \\
evplot & visualisation tool for evolution plots (energy vs time etc) using
PGPLOT
\end{tabular}
Type
\begin{verbatim}
make install
\end{verbatim}
or simply `make' to compile the whole
code. 

\subsection{Running a job}
\subsubsection{Creating an input file}

\subsubsection{Rerunning a job from the last position}

\subsection{Input options}

\subsection{Test problems}
 A suite of test problems

\section{Setting up a simulation}

\subsection{Using the interactive setup}

\subsection{Writing your own particle setup}

\section{The Gory details}

\subsection{Neighbour finding}

\subsection{Time stepping}

\section{Common errors}

\section{Visualisation using supersphplot}

\section{Wishlist for future improvements}

\section*{Acknowledgements}

\end{document}
